\chapter{Global Declarations\label{chap:GlobalDeclarations}}
\hypertarget{def-globaldeclarationterm}{}
Global declarations are grammatically derived from $\Ndecl$ and represented as ASTs by $\decl$.

There are four kinds of global declarations:
\begin{itemize}
  \item Subprogram declarations, defined in \chapref{SubprogramDeclarations};
  \item Type declarations, defined in \chapref{TypeDeclarations};
  \item Global storage declarations, defined in \chapref{GlobalStorageDeclarations};
  \item Global pragmas.
\end{itemize}

The typing of global declarations is defined in \secref{GlobalDeclarationsTyping}.
As the only kind of global declarations that are associated with semantics are global storage declarations,
their semantics is given in \secref{GlobalStorageDeclarationsSemantics}.

Global pragmas are statically checked by the typechecker, but do not produce \typedast{} nodes,
and thus are not associated with a dynamic semantics.

For example, the specification in \listingref{GlobalPragma} contains two global pragmas,
which are ignored by the dynamic semantics.
\ASLListing{Global pragmas}{GlobalPragma}{\definitiontests/GlobalPragma.asl}

It is recommended that tools which process ASL should warn users of unrecognized tool specific pragmas.
\identi{SLNQ}

For example, a tool processing the specification in \listingref{GlobalPragma2} may recognize \verb|my_tool_pragma1|
but not \verb|other_tool_op| and warn about it.
\ASLListing{Warning of pragmas not recognized by tools}{GlobalPragma2}{\definitiontests/GlobalPragma2.asl}

%%%%%%%%%%%%%%%%%%%%%%%%%%%%%%%%%%%%%%%%%%%%%%%%%%%%%%%%%%%%%%%%%%%%%%%%%%%%%%%%
\section{Syntax}
%%%%%%%%%%%%%%%%%%%%%%%%%%%%%%%%%%%%%%%%%%%%%%%%%%%%%%%%%%%%%%%%%%%%%%%%%%%%%%%%
Subprogram declarations:
\begin{flalign*}
\Ndecl  \derives \ & \Nqualifier \parsesep \Noverride \parsesep \Tfunc \parsesep \Tidentifier \parsesep \Nparamsopt \parsesep \Nfuncargs \parsesep \Nreturntype \parsesep &\\
    &\wrappedline \Nrecurselimit \parsesep \Nfuncbody\\
|\ & \Nqualifier \parsesep \Noverride \parsesep \Tfunc \parsesep \Tidentifier \parsesep \Nparamsopt \parsesep \Nfuncargs \parsesep \Nfuncbody &\\
|\ & \Noverride \parsesep \Taccessor \parsesep \Tidentifier \parsesep \Nparamsopt \parsesep \Nfuncargs \parsesep \Tbeq \parsesep \Tidentifier \parsesep \Nasty &\\
   & \wrappedline\ \Naccessorbody &\\
\Naccessorbody \derives \ & \Tbegin \parsesep \Naccessors \parsesep \Tend \parsesep \Tsemicolon&
\end{flalign*}

Type declarations:
\begin{flalign*}
\Ndecl  \derives \ & \Ttype \parsesep \Tidentifier \parsesep \Tof \parsesep \Ntydecl \parsesep \Nsubtypeopt \parsesep \Tsemicolon&\\
|\ & \Ttype \parsesep \Tidentifier \parsesep \Nsubtype \parsesep \Tsemicolon&\\
\end{flalign*}

Global storage declarations:
\begin{flalign*}
\Ndecl  \derives \ & \Nglobaldeclkeywordnonconfig \parsesep \Tidentifier \parsesep &\\
    & \wrappedline\ \option{\Tcolon \parsesep \Nty} \parsesep \Teq \Nexpr \parsesep \Tsemicolon &\\
|\ & \Tconfig \parsesep \Tidentifier \parsesep \Tcolon \parsesep \Nty \parsesep \Teq \parsesep \Nexpr \parsesep \Tsemicolon &\\
|\ & \Tvar \parsesep \Tidentifier \parsesep \Tcolon \parsesep \Nty \parsesep \Tsemicolon&\\
|\ & \Nglobaldeclkeyword \parsesep \Tidentifier \parsesep \Tcolon \parsesep \Nty &\\
   & \wrappedline\ \Teq \parsesep \Nelidedparamcall \parsesep \Tsemicolon &\\
\end{flalign*}

Pragmas:
\begin{flalign*}
\Ndecl  \derives \ & \Tpragma \parsesep \Tidentifier \parsesep \ClistZero{\Nexpr} \parsesep \Tsemicolon&
\end{flalign*}

%%%%%%%%%%%%%%%%%%%%%%%%%%%%%%%%%%%%%%%%%%%%%%%%%%%%%%%%%%%%%%%%%%%%%%%%%%%%%%%%
\section{Abstract Syntax}
%%%%%%%%%%%%%%%%%%%%%%%%%%%%%%%%%%%%%%%%%%%%%%%%%%%%%%%%%%%%%%%%%%%%%%%%%%%%%%%%
\begin{flalign*}
\decl \derives\ & \DFunc(\func) & \\
  |\ & \DGlobalStorage(\globaldecl) & \\
  |\ & \DTypeDecl(\Tidentifier, \ty, (\Tidentifier, \overtext{\Field^*}{with fields})?) & \\
  |\ & \DPragma(\Tidentifier, \overtext{\expr^*}{args}) &
\end{flalign*}

\hypertarget{build-decl}{}
The relation
\[
  \builddecl : \overname{\parsenode{\Ndecl}}{\vparsednode} \;\aslrel\; \overname{\decl^*}{\vastnode}
\]
transforms a parse node $\vparsednode$ into an AST node $\vastnode$.

The case rules for building global declarations are the following:
\begin{itemize}
  \item \ASTRuleRef{GlobalStorageDecl} for global storage declarations
  \item \ASTRuleRef{TypeDecl} for type declarations
  \item \ASTRuleRef{GlobalPragma} for global pragmas
\end{itemize}

\ASTRuleDef{GlobalPragma}
\begin{mathpar}
\inferrule[global\_pragma]{
  \buildclist[\Nexpr](\vargs) \astarrow \astversion{\vargs}
}{
  {
  \begin{array}{r}
  \builddecl(\overname{\Ndecl(\Tpragma, \Tidentifier(\id), \namednode{\vargs}{\ClistZero{\Nexpr}}, \Tsemicolon)}{\vparsednode})
  \astarrow \\
    \overname{\left[\DPragma(\id, \astversion{\vargs})\right]}{\vastnode}
  \end{array}
  }
}
\end{mathpar}

%%%%%%%%%%%%%%%%%%%%%%%%%%%%%%%%%%%%%%%%%%%%%%%%%%%%%%%%%%%%%%%%%%%%%%%%%%%%%%%%
\section{Typing Global Declarations\label{sec:GlobalDeclarationsTyping}}
%%%%%%%%%%%%%%%%%%%%%%%%%%%%%%%%%%%%%%%%%%%%%%%%%%%%%%%%%%%%%%%%%%%%%%%%%%%%%%%%
\hypertarget{def-typecheckdecl}{}
The function
\[
  \typecheckdecl(
    \overname{\globalstaticenvs}{\genv} \aslsep
    \overname{\decl}{\vd}
  )
  \aslto (\overname{\decl}{\newd} \times \overname{\globalstaticenvs}{\newgenv})
  \cup \overname{\TTypeError}{\TypeErrorConfig}
\]
annotates a global declaration $\vd$ in the global static environment $\genv$,
yielding an annotated global declaration $\newd$ and modified global static environment $\newgenv$.
\ProseOtherwiseTypeError

\TypingRuleDef{TypecheckDecl}
\ExampleDef{Typing Global Declarations}
\listingref{TypecheckDecl} exemplifies various kinds of global declarations ---
types, global storage elements, and subprograms.
All global declarations in \listingref{TypecheckDecl} are well-typed.
\ASLListing{Typing global declarations}{TypecheckDecl}{\typingtests/TypingRule.TypecheckDecl.asl}

\ProseParagraph
\OneApplies
\begin{itemize}
  \item \AllApplyCase{func}
  \begin{itemize}
    \item $\vd$ is a subprogram AST node with a subprogram definition $\vf$, that is, $\DFunc(\vf)$;
    \item annotating and declaring the subprogram for $\vf$ in $\genv$ as per \\
          \TypingRuleRef{AnnotateAndDeclareFunc}
          yields the environment $\tenvone$, a subprogram definition $\vfone$,
          and a \sideeffectdescriptorsetsterm{} $\vsesfuncsig$\ProseOrTypeError;
    \item \Proseannotatesubprogram{$\tenv$}{$\vfone$}{\vsesfuncsig}{$\newf$}{$\vsesf$}\ProseOrTypeError;
    \item applying $\addsubprogram$ to $\tenvone$, $\newf.\funcname$, $\newf$, and $\vsesf$ yields $\newtenv$;
    \item define $\newd$ as the subprogram AST node with $\newf$, that is, $\DFunc(\newf)$;
    \item define $\newgenv$ as the global component of $\newtenv$.
  \end{itemize}

  \item \AllApplyCase{global\_storage}
  \begin{itemize}
    \item $\vd$ is a global storage declaration with description $\gsd$, that is, \\ $\DGlobalStorage(\gsd)$;
    \item declaring the global storage with description $\gsd$ in $\genv$ yields the new environment
          $\newgenv$ and new global storage description $\gsdp$\ProseOrTypeError;
    \item define $\newd$ as the global storage declaration with description $\gsdp$, that is, \\ $\DGlobalStorage(\gsdp)$.
  \end{itemize}

  \item \AllApplyCase{type}
  \begin{itemize}
    \item $\vd$ is a type declaration with identifier $\vx$, type $\tty$,
          and \optional\ field initializers $\vs$, that is, $\DTypeDecl(\vx, \tty, \vs)$;
    \item declaring the type described by $(\vx, \tty, \vs)$ in $\genv$
          as per \\
          \TypingRuleRef{DeclaredType} yields the modified global static environment \\
          $\newgenv$\ProseOrTypeError;
    \item define $\newd$ as $\vd$.
  \end{itemize}
\end{itemize}

\FormallyParagraph
\begin{mathpar}
\inferrule[func]{
  \annotateanddeclarefunc(\genv, \vf) \typearrow (\tenvone, \vfone, \vsesfuncsig) \OrTypeError\\\\
  \annotatesubprogram(\tenvone, \vfone, \vsesfuncsig) \typearrow (\newf,\vsesf) \OrTypeError\\\\
  \addsubprogram(\tenvone, \newf.\funcname, \newf, \vsesf) \typearrow \newtenv
}{
  \typecheckdecl(\genv, \overname{\DFunc(\vf)}{\vd})
  \typearrow (\overname{\DFunc(\newf)}{\newd}, \overname{G^\newtenv}{\newgenv})
}
\end{mathpar}

\begin{mathpar}
\inferrule[global\_storage]{
  \declareglobalstorage(\genv, \gsd) \typearrow (\newgenv, \gsdp) \OrTypeError
}{
  {
    \begin{array}{r}
  \typecheckdecl(\genv, \overname{\DGlobalStorage(\gsd)}{\vd})
  \typearrow \\ (\overname{\DGlobalStorage(\gsdp)}{\newd}, \newgenv)
    \end{array}
  }
}
\end{mathpar}

\begin{mathpar}
\inferrule[type]{
  \declaretype(\genv, \vx, \tty, \vs) \typearrow \newgenv \OrTypeError
}{
  \typecheckdecl(\genv, \overname{\DTypeDecl(\vx, \tty, \vs)}{\vd}) \typearrow (\overname{\vd}{\newd}, \newgenv)
}
\end{mathpar}
\CodeSubsection{\TypecheckDeclBegin}{\TypecheckDeclEnd}{../Typing.ml}

\TypingRuleDef{Subprogram}
\hypertarget{def-annotatesubprogram}{}
The function
\[
\begin{array}{r}
  \annotatesubprogram(\overname{\staticenvs}{\tenv} \aslsep \overname{\func}{\vf} \aslsep \overname{\TSideEffectSet}{\vsesfuncsig})
  \aslto \\
  (\overname{\func}{\vfp} \times \overname{\TSideEffectSet}{\vses})\ \cup \overname{\TTypeError}{\TypeErrorConfig}
\end{array}
\]
annotates a subprogram $\vf$ in an environment $\tenv$ and \sideeffectsetterm\ $\vsesfuncsig$, resulting in an annotated subprogram $\vfp$
and inferred \sideeffectsetterm\ $\vses$.
\ProseOtherwiseTypeError

Note that the return type in $\vf$ has already been annotated by $\annotatefuncsig$.

\ExampleDef{Annotating Subprograms}
\listingref{typing-subprogram} shows an example of a well-typed procedure --- \verb|my_procedure|
and an example of a well-typed function --- \verb|flip_bits|.
\ASLListing{Typing subprograms}{typing-subprogram}{\typingtests/TypingRule.Subprogram.asl}

\ProseParagraph
\AllApply
\begin{itemize}
  \item annotating $\vf.\funcbody$ in $\tenv$ as per \TypingRuleRef{Block} yields \\
        $(\newbody, \vsesbody)$\ProseOrTypeError;
  \item applying $\checkcontrolflow$ to $\tenv$, $\vf$, and $\newbody$ yields $\True$\ProseOrTypeError;
  \item $\vfp$ is $\vf$ with the subprogram body substituted with $\newbody$;
  \item \Proseeqdef{$\vsesp$}{the union of $\vsesfuncsig$ and $\vsesbody$ with every \LocalEffectTerm{} removed};
  \item checking that the \sideeffectsetterm{} $\vsesp$ adheres to the subprogram's \purity{} qualifier $\vf.\funcqualifier$ yields $\True$\ProseOrTypeError;
  \item the \sideeffectsetterm{} corresponding to $\vf.\funcqualifier$ (via \\
        $\sesforsubprogram$) is $\vses$.
\end{itemize}

\FormallyParagraph
\begin{mathpar}
\inferrule{
  \annotateblock{\tenv, \vf.\funcbody} \typearrow (\newbody, \vsesbody) \OrTypeError\\\\
  \checkcontrolflow(\tenv, \vf, \newbody) \typearrow \True \OrTypeError\\\\
  \vfp \eqdef \substrecordfield(\vf, \funcbody, \newbody)\\
  \vsesp \eqdef \vsesfuncsig \cup (\vsesbody \setminus \{ \vs \;|\; \configdomain{\vs} = \LocalEffect \}) \\
  \checksubprogrampurity(\vf.\funcqualifier, \vsesp) \typearrow \True \OrTypeError \\
  \sesforsubprogram(\vf.\funcqualifier) \typearrow \vses
}{
  \annotatesubprogram(\tenv, \vf, \vsesfuncsig) \typearrow (\vfp, \vses)
}
\end{mathpar}

\identi{GHGK} \identr{HWTV} \identr{SCHV} \identr{VDPC}
\identr{TJKQ} \identi{LFJZ} \identi{BZVB} \identi{RQQB}
\CodeSubsection{\SubprogramBegin}{\SubprogramEnd}{../Typing.ml}

\TypingRuleDef{CheckControlFlow}
\hypertarget{def-checkcontrolflow}{}
The function
\[
  \checkcontrolflow(
    \overname{\staticenvs}{\tenv} \aslsep
    \overname{\func}{\vf} \aslsep
    \overname{\stmt}{\vbody})
  \typearrow \{\True\} \cup \overname{\TTypeError}{\TypeErrorConfig}
\]
checks whether the annotated body statement $\vs$ of the function definition $\vf$
obeys the control-flow requirements in the static environment $\tenv$.

\ExampleDef{Ensuring All Terminating Paths Terminate Correctly}
In \listingref{CheckControlFlow}, every evaluation of the function body for\\
\verb|all_terminating_paths_correct| terminates by either returning
a value, throwing an exception, or evaluating an \unreachablestatementterm.
\ASLListing{All terminating paths terminate correctly}{CheckControlFlow}{\typingtests/TypingRule.CheckControlFlow.asl}

In \listingref{CheckControlFlow-bad2}, the path through the function body
for \verb|incorrect_terminating_path|
where \verb|v != Zeros{N}| evaluates to
$\True$ and \verb|flag| evaluates to $\False$ terminates without
returning a value, throwing an exception, or evaluating an \unreachablestatementterm,
which is a \typingerrorterm.
\ASLListing{An incorrectly terminating path}{CheckControlFlow-bad2}{\typingtests/TypingRule.CheckControlFlow.bad2.asl}

\ExampleDef{Illegally returning from a \texttt{noreturn} Subprogram}
The specification in \listingref{CheckControlFlow-bad}
is ill-typed as \verb|returning| is declared with the \verb|noreturn|
qualifier, yet it contains the \returnstatementsterm{} \verb|return 0;|.
\ASLListing{Illegally returning from a \texttt{noreturn} subprogram}{CheckControlFlow-bad}{\typingtests/TypingRule.CheckControlFlow.bad.asl}

\ProseParagraph
\AllApply
\begin{itemize}
  \item applying $\approxstmt$ to $\tenv$ and $\vbody$ yields $\absconfigs$;
  \item checking that $\absconfigs$ is a subset of the set of \Proseabstractconfigurations{}
        allowed for $\vf$, as determined by $\allowedabsconfigs$ yields $\True$\ProseTerminateAs{\BadSubprogramDeclaration}.
  \item the result is $\True$.
\end{itemize}

\FormallyParagraph
\begin{mathpar}
\inferrule{
  \approxstmt(\tenv, \vbody) \typearrow \absconfigs\\
  \checktrans{\absconfigs \subseteq \allowedabsconfigs(\vf)}{\BadSubprogramDeclaration} \typearrow \True \OrTypeError
}{
  \checkcontrolflow(\tenv, \vf, \vbody) \typearrow \True
}
\end{mathpar}

\TypingRuleDef{AllowedAbsConfigs}
\hypertarget{def-allowedabsconfigs}{}
The function
\[
  \allowedabsconfigs(\overname{\func}{\vf})
  \typearrow \overname{\pow{\{\AbsContinuing, \AbsReturning, \AbsAbnormal\}}}{\absconfigs}
\]
determines the set of \Proseabstractconfigurations{} allowed for the function definition $\vf$,
yielding the result in $\absconfigs$.

\ExampleDef{The Allowed Abstract Configurations for a Subprogram}
The set of \Proseabstractconfigurations{} allowed for the subprogram \verb|all_terminating_paths_correct|
in \listingref{CheckControlFlow} is $\{\AbsAbnormal, \AbsReturning\}$, since it is
a function.

The set of \Proseabstractconfigurations{} allowed for the subprogram \verb|returning|
is in \listingref{CheckControlFlow-bad} is $\{\AbsAbnormal\}$, since it is qualified
by \verb|noreturn|.

\ProseParagraph
\OneApplies
\begin{itemize}
  \item \AllApplyCase{noreturn}
  \begin{itemize}
    \item $\vf$ is qualified by \verb|noreturn|;
    \item \Proseeqdef{$\absconfigs$}{the set consisting of $\AbsContinuing$ and \\
          $\AbsAbnormal$}.
  \end{itemize}

  \item \AllApplyCase{returning\_proc}
  \begin{itemize}
    \item $\vf$ is not qualified by \verb|noreturn|;
    \item $\vf$ is a procedure;
    \item \Proseeqdef{$\absconfigs$}{the set consisting of $\AbsContinuing$, $\AbsReturning$, $\AbsAbnormal$}.
  \end{itemize}

  \item \AllApplyCase{func}
  \begin{itemize}
    \item $\vf$ is as function;
    \item \Proseeqdef{$\absconfigs$}{the set consisting of $\AbsReturning$, $\AbsAbnormal$}.
  \end{itemize}
\end{itemize}

\FormallyParagraph
\begin{mathpar}
\inferrule[noreturn]{
  \vf.\funcqualifier = \Some{\Noreturn}
}{
  \allowedabsconfigs(\vf) \typearrow \overname{\{\AbsAbnormal\}}{\absconfigs}
}
\end{mathpar}

\begin{mathpar}
\inferrule[returning\_proc]{
  \vf.\funcqualifier \neq \Some{\Noreturn} \land \vf.\funcreturntype = \None
}{
  \allowedabsconfigs(\vf) \typearrow \overname{\{\AbsContinuing, \AbsReturning, \AbsAbnormal\}}{\absconfigs}
}
\end{mathpar}

\begin{mathpar}
\inferrule[func]{
  \vf.\funcreturntype \neq \None
}{
  \allowedabsconfigs(\vf) \typearrow \overname{\{\AbsReturning, \AbsAbnormal\}}{\absconfigs}
}
\end{mathpar}

\TypingRuleDef{ApproxStmt}
\hypertarget{def-approxstmt}{}
The function
\[
  \approxstmt(\overname{\staticenvs}{\tenv} \aslsep \overname{\stmt}{\vs})
  \typearrow\overname{\{\AbsContinuing, \AbsReturning, \AbsAbnormal\}}{\absconfigs}
\]
returns in $\absconfigs$ a superset of the set of \Proseabstractconfigurations{} (defined next),
that an evaluation of $\vs$ in any environment consisting of the static environment $\tenv$ yields.

\hypertarget{def-abstractconfigurations}{}
\hypertarget{def-absabnormal}{}
\hypertarget{def-absreturning}{}
\hypertarget{def-abscontinuing}{}
The set of \Proseabstractconfigurations{} consists of symbols that represent
subsets of the set of dynamic configurations that may be returned by
$\evalstmt{(\tenv, \cdot), \vs}$:
\begin{itemize}
  \item $\AbsContinuing$ represents $\TContinuing$;
  \item $\AbsReturning$ represents $\TReturning$; and
  \item $\AbsAbnormal$ represents $\TThrowing \cup \TDynError$.
\end{itemize}

\ExampleDef{Determining the Abstract Configurations of Statements}
In \listingref{ApproxStmt}, the function bodies of all functions
terminate by either returning a value, throwing an exception, or executing
the \unreachablestatementterm.
\ASLListing{Determining the abstract configurations of statements}{ApproxStmt}{\typingtests/TypingRule.ApproxStmt.asl}

In \listingref{ApproxStmt-bad1}, the function body of \verb|loop_forever|
is ill-typed, since the conservative analysis of \TypingRuleRef{ApproxStmt}
cannot determine that the \whilestatementterm{} never terminates.
To make the function body well-typed, another statement following the loop
can be added, for example, an \unreachablestatementterm.
\ASLListing{An ill-typed function body}{ApproxStmt-bad1}{\typingtests/TypingRule.ApproxStmt.bad1.asl}

\ProseParagraph
\OneApplies
\begin{itemize}
  \item \AllApplyCase{s\_pass}
  \begin{itemize}
    \item $\vs$ is a \passstatementterm;
    \item \Proseeqdef{$\absconfigs$}{the singleton set for $\AbsContinuing$}.
  \end{itemize}

  \item \AllApplyCase{simple}
  \begin{itemize}
    \item $\vs$ is either a \declarationstatementterm, an \assignmentstatementterm,
          an \assertionstatementterm, or a \printstatementsterm;
    \item \Proseeqdef{$\absconfigs$}{the set consisting of $\AbsContinuing$ and \\
          $\AbsAbnormal$}.
  \end{itemize}

  \item \AllApplyCase{s\_unreachable}
  \begin{itemize}
    \item $\vs$ is the \unreachablestatementterm;
    \item \Proseeqdef{$\absconfigs$}{the singleton set for $\AbsAbnormal$}.
  \end{itemize}

  \item \AllApplyCase{s\_call}
  \begin{itemize}
    \item $\vs$ is a \callstatementterm{} consisting of the call record $\call$, that is, $\SCall(\call)$;
    \item \Proseeqdef{$\vf$}{the $\func$ AST node bound to $\call.\callname$ in the global static environment
          map $\subprograms$};
    \item \Proseeqdef{$\absconfigs$}{the singleton set for $\AbsAbnormal$, if $\vf$ has the\\
          \texttt{noreturn} qualifier, and the set consisting of $\AbsContinuing$ and \\
          $\AbsAbnormal$ otherwise}.
  \end{itemize}

  \item \AllApplyCase{s\_return}
  \begin{itemize}
    \item $\vs$ is a \returnstatementsterm;
    \item \Proseeqdef{$\absconfigs$}{the set consisting of $\AbsReturning$ and $\AbsAbnormal$}.
  \end{itemize}

  \item \AllApplyCase{s\_throw}
  \begin{itemize}
    \item $\vs$ is a \throwstatementsterm;
    \item  \Proseeqdef{$\absconfigs$}{the singleton set for $\AbsAbnormal$}.
  \end{itemize}

  \item \AllApplyCase{s\_seq}
  \begin{itemize}
    \item $\vs$ is the \sequencingstatement{$\vsone$}{$\vstwo$};
    \item applying $\approxstmt$ to $\tenv$ and $\vsone$ yields $\configsone$;
    \item applying $\approxstmt$ to $\tenv$ and $\vstwo$ yields $\configstwo$;
    \item for each \Proseabstractconfiguration{} $\vc$ in $\configsone$ \Proseeqdef{$\vs_\vc$}{
      $\configstwo$ if $\vc$ is $\AbsContinuing$ and the singleton set for $\vc$ otherwise};
    \item \Proseeqdef{$\absconfigs$}{the union of the sets $\vs_\vc$ for each \Proseabstractconfiguration{} $\vc$ in $\configsone$}.
  \end{itemize}

  \item \AllApplyCase{loop}
  \begin{itemize}
    \item $\vs$ matches one of the following statements: a \repeatstatementsterm{} with body statement $\vbody$,
          a \forstatementterm{} with body statement $\vbody$, or a \whilestatementterm{} with body statement $\vbody$;
    \item applying $\approxstmt$ to $\tenv$ and $\vbody$ yields $\bodyconfigs$;
    \item \Proseeqdef{$\absconfigs$}{the union of $\bodyconfigs$ and $\AbsAbnormal$}.
  \end{itemize}

  \item \AllApplyCase{s\_cond}
  \begin{itemize}
    \item $\vs$ is a \conditionalstatement{$\Ignore$}{$\vsone$}{$\vstwo$};
    \item applying $\approxstmt$ to $\tenv$ and $\vsone$ yields $\configsone$;
    \item applying $\approxstmt$ to $\tenv$ and $\vstwo$ yields $\configstwo$;
    \item \Proseeqdef{$\absconfigs$}{the union of $\configsone$, $\configstwo$, and $\AbsAbnormal$}.
  \end{itemize}

  \item \AllApplyCase{s\_try}
  \begin{itemize}
    \item $\vs$ is the \trystatement{$\vbody$}{\\
          $\catchers$}{$\otherwise$};
    \item applying $\approxstmt$ to $\tenv$ and $\vbody$ yields $\bodyconfigs$;
    \item \Proseeqdef{$\otherwiseconfigs$}{the application of $\approxstmt$ to $\tenv$ and $\votherwise$,
          if $\votherwise$ is the optional containing $\otherwises$ and the empty set otherwise};
    \item for every statement $\vc$ appearing in a catcher in the list $\catchers$, applying $\approxstmt$ to $\tenv$ and $\vc$
          yields $\catchconfigs_\vc$;
    \item \Proseeqdef{$\absconfigs$}{the union of $\bodyconfigs$, $\otherwiseconfigs$,
          and  $\catchconfigs_\vc$, for every statement $\vc$ appearing in a catcher in the list $\catchers$}.
  \end{itemize}
\end{itemize}

\FormallyParagraph
\begin{mathpar}
\inferrule[s\_pass]{}{
  \approxstmt(\tenv, \overname{\SPass}{\vs}) \typearrow \overname{\{\AbsContinuing\}}{\absconfigs}
}
\end{mathpar}

\begin{mathpar}
\inferrule[simple]{
  \configdomain{\vs} \in \{\SDecl, \SAssign, \SAssert, \SPrint\}
}{
  \approxstmt(\tenv, \vs) \typearrow \overname{\{\AbsContinuing, \AbsAbnormal\}}{\absconfigs}
}
\end{mathpar}

\begin{mathpar}
\inferrule[s\_unreachable]{}{
  \approxstmt(\tenv, \overname{\SUnreachable}{\vs}) \typearrow \overname{\{\AbsAbnormal \}}{\absconfigs}
}
\end{mathpar}

\begin{mathpar}
\inferrule[s\_call]{
  G^\tenv.\subprograms(\call.\callname) = (\vf, \Ignore)\\
  {
  \absconfigs \eqdef
  \begin{cases}
    \{\AbsAbnormal\} & \text{if }\vf.\funcqualifier = \Some{\Noreturn}\\
    \{\AbsContinuing, \AbsAbnormal\} & \text{else}
  \end{cases}
  }
}{
  \approxstmt(\tenv, \overname{\SCall(\call)}{\vs}) \typearrow \absconfigs
}
\end{mathpar}

\begin{mathpar}
\inferrule[s\_return]{}{
  \approxstmt(\tenv, \overname{\SReturn(\Ignore)}{\vs}) \typearrow \overname{\{\AbsReturning, \AbsAbnormal \}}{\absconfigs}
}
\end{mathpar}

\begin{mathpar}
\inferrule[s\_throw]{}{
  \approxstmt(\tenv, \overname{\SThrow(\Ignore)}{\vs}) \typearrow \overname{\{\AbsAbnormal \}}{\absconfigs}
}
\end{mathpar}

\begin{mathpar}
\inferrule[s\_seq]{
  \approxstmt(\tenv, \vsone) \typearrow \configsone\\
  \approxstmt(\tenv, \vstwo) \typearrow \configstwo\\
  \vc\in\configsone: \vs_\vc \eqdef \choice{\vc = \AbsContinuing}{\configstwo}{\{\vc\}}\\
  \absconfigs \eqdef \bigcup_{\vc\in\configsone} \vs_\vc
}{
  \approxstmt(\tenv, \overname{\SSeq(\vsone, \vstwo)}{\vs}) \typearrow \absconfigs
}
\end{mathpar}

\begin{mathpar}
\inferrule[loop]{
  {
  \left(
  \begin{array}{rcll}
    \vs &=& \SRepeat(\vbody, \Ignore, \Ignore) &\lor\\
    \vs &=& \SFor \{ \Forbody : \vbody, \ldots \} &\lor\\
    \vs &=& \SWhile(\Ignore, \Ignore, \vbody) &
  \end{array}
  \right)
  }\\
  \approxstmt(\tenv, \vbody) \typearrow \bodyconfigs\\
}{
  \approxstmt(\tenv, \overname{\SCond(\Ignore, \vsone, \vstwo)}{\vs}) \typearrow \overname{\bodyconfigs \cup \{\AbsAbnormal\}}{\absconfigs}
}
\end{mathpar}

\begin{mathpar}
\inferrule[s\_cond]{
  \approxstmt(\tenv, \vsone) \typearrow \configsone\\
  \approxstmt(\tenv, \vstwo) \typearrow \configstwo
}{
  \approxstmt(\tenv, \overname{\STry(\vbody, \vsone, \vstwo)}{\vs}) \typearrow \overname{\{\AbsAbnormal\} \cup \vsone \cup \vstwo}{\absconfigs}
}
\end{mathpar}

\begin{mathpar}
\inferrule[s\_try]{
  \approxstmt(\vbody, \vsone) \typearrow \bodyconfigs\\
  {
    \otherwiseconfigs \eqdef \left\{
    \begin{array}{ll}
    \textbf{if}   & \votherwise=\Some{\otherwises}\textbf{ then}\\
                  & \approxstmt(\tenv, \otherwises)\\
    \textbf{else} & \emptyset
    \end{array}\right.
  }\\
  (\Ignore,\Ignore,\vc)\in\catchers: \approxstmt(\tenv, \vc) \typearrow \catchconfigs_\vc\\
  {
  \absconfigs \eqdef \left\{
  \begin{array}{ll}
  \bodyconfigs & \cup\\
  \otherwiseconfigs & \cup\\
  \bigcup_{(\Ignore,\Ignore,\vc)\in\catchers} \catchconfigs_\vc &
  \end{array}\right.
  }
}{
  \approxstmt(\tenv, \overname{\STry(\vbody, \catchers, \votherwise)}{\vs}) \typearrow \absconfigs
}
\end{mathpar}
